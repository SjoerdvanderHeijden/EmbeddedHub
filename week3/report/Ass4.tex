\documentclass{article}

\usepackage[hidelinks]{hyperref}
\usepackage[utf8]{inputenc}
%\usepackage[latin1]{inputenc}
%\documentclass[15pt,a4paper]{article}
%\usepackage[latin1]{inputenc}
\usepackage{amsmath}
\usepackage{amsfonts}
\usepackage{amssymb}
\usepackage{amsthm}
\usepackage[titletoc,title]{appendix}
\usepackage[english]{babel}
\usepackage[font= small,format=plain,labelfont=bf,up,textfont=it,up]{caption}
\usepackage{colortbl}
\usepackage{diagbox}
\usepackage{eurosym, placeins}
\usepackage{fullpage,graphicx,textcomp,float,gensymb,wrapfig}
\usepackage{longtable}
\usepackage{makecell}
\usepackage{multicol}
\usepackage{mathrsfs}
\usepackage[numbers]{natbib}
\usepackage{parskip}
\usepackage[detect-all]{siunitx}
\usepackage{subcaption}
\usepackage{tabularx}
\newtheorem{theorem}{Theorem}
\usepackage[colorinlistoftodos]{todonotes}
\usepackage{wasysym}


\newcommand{\HRule}{\rule{\linewidth}{0.5mm}}
\newcommand{\red}[1]{{\color{red}{#1}}}

\usepackage{textcomp} % for a single apostrophe with \textquotesingle

\edef\restoreparindent{\parindent=\the\parindent\relax}
\restoreparindent

\allowdisplaybreaks

\graphicspath{{../}}

\sisetup{range-phrase=--}
\sisetup{range-units=single}

\newcommand{\subfigimg}[3][,]{%
	\setbox1=\hbox{\includegraphics[#1]{#3}}% Store image in box
	\leavevmode\rlap{\usebox1}% Print image
	\rlap{\hspace*{0.9\linewidth}\raisebox{\dimexpr\ht1-1.75\baselineskip}{#2}}% Print label
	\phantom{\usebox1}% Insert appropriate spacing
}



\title{Embedded systems: Assignment 4}
\date{\today}
\author{Sjoerd van der Heijden \\ 10336001}

\begin{document}
	\maketitle
%	\tableofcontents
	\pagenumbering{arabic}
	\section*{Introduction}
	\par In this document I will show and explain my solutions to the tasks detailed in Assignment 4 of the course Embedded Systems 2018.
	
	\section*{Assignment 4.1}
	\par For this assignment I had to analyze a Petri net (created using PIPE 5 \cite{dingle2009pipe2, bonet2007pipe}) and find potential deadlocks. I recognize the net as a loose implementation of assignment 3.1. A proper implementation of that assignment would in fact have the possibility of going into deadlock, namely when there are cars in every Arrive node, as then all cars should give priority to the car on their right while never taking priority. However, compared to such an implementation, this implementation lacks an inhibitor going from Arrive2 to event T5, thus allowing cars in loop 3 to cross regardless of anything happening in loop 2. 
	\par What this means is that events in loop 3 are the ``hardest'' to stop from firing,	but for the Petri net to go into deadlock, even those events must become impossible to fire. This can only happen if there would be no tokens in nodes Excl 1 or 2 and no way to put any in them, and this only happens if cars in loop 0 or 2 are crossing. However, the event to fill aforementioned nodes may then alway fire, allowing events in loop 3 to fire again.
	\par Ergo, no deadlocks can occur in this Petri net.
	
	
	\section*{Assignment 4.2}
	\par The tasks here was to finish the tables shown in the lecture. Tables  \ref{tab:subtraction}, \ref{tab:multiplication} and \ref{tab:division} show such tables for the subtraction, multiplication and division operations on the abstract domain $\{-,0,+\}$, repspectively. Funnily enough, the tables for commutative operators are symmetric along the diagonal.
	\par Note that for division, I have considered integer division, meaning that $ \{ + \} /^\# \{ + \}$ results in $\{0,+\}$, since $1/1000=0$ according to some. The same happens for any combination between $+$ and $-$, as for completeness I assume there may be languages where $-1/1000=0$.
	
	
	\begin{table}[ht!]
		\caption{Complete table of the subtraction operation on the abstract domain $\{-,0,+\}$.}\label{tab:subtraction}
		\begin{tabular}{>{\columncolor[gray]{0.8}}l||l|l|l|l|l|l|l}
			\rowcolor[gray] {.8}
			$-^\#$ & \{0\} & \{+\} & \{-\} & \{-,0\} & \{-,+\} & \{0,+\} & \{-,0,+\} \\\hline\hline
			%\rowcolor[gray] {.8}
			\{0\} & \{0\} & \{-\} & \{+\} & \{0,+\} & \{-,+\} & \{-,0\} & \{-,0,+\} \\
			
			\{+\} & \{+\} & \{-,0,+\} & \{+\} & \{+\} & \{-,0,+\} & \{-,0,+\} & \{-,0,+\} \\
			
			%\rowcolor[gray] {.8}
			\{-\} & \{-\} & \{-\} & \{-,0,+\} & \{-,0,+\} & \{-,0,+\} & \{-\} & \{-,0,+\} \\
			
			\{-,0\} & \{-,0\} & \{-\} & \{-,0,+\} & \{-,0,+\} & \{-,0,+\} & \{-,0\} & \{-,0,+\} \\
			
			%\rowcolor[gray] {.8}
			\{-,+\} & \{-,+\} & \{-,0,+\} & \{-,0,+\} & \{-,0,+\} & \{-,0,+\} & \{-,0,+\} & \{-,0,+\} \\
			
		  	\{0,+\} & \{0,+\} & \{-,0,+\} & \{+\} & \{0,+\} & \{-,0,+\} & \{-,0,+\} & \{-,0,+\} \\
		  	
			%\rowcolor[gray] {.8}
			\{-,0,+\} & \{-,0,+\} & \{-,0,+\} & \{-,0,+\} & \{-,0,+\} & \{-,0,+\} & \{-,0,+\} & \{-,0,+\} \\
		\end{tabular}
	\end{table}


	\begin{table}[ht!]
		\caption{Complete table of the multiplication operation on the abstract domain $\{-,0,+\}$.}\label{tab:multiplication}
		\begin{tabular}{>{\columncolor[gray]{0.8}}l||l|l|l|l|l|l|l}
			\rowcolor[gray] {.8}
			$*^\#$	& \{0\}	& \{+\}		& \{-\}		& \{-,0\}	& \{-,+\}	& \{0,+\} & \{-,0,+\} \\\hline\hline
			%\rowcolor[gray] {.8}
			\{0\}	& \{0\}	& \{0\}		& \{0\}		& \{0\}		& \{0\}		& \{0\}	& \{0\} \\
			
			\{+\} 	& \{0\}	& \{+\}		& \{-\}		& \{-,0\}	& \{-,+\}	& \{0,+\} & \{-,0,+\} \\
			
			%\rowcolor[gray] {.8}
			\{-\}	& \{0\}	& \{-\}		& \{+\}		& \{0,+\}	& \{-,+\}	& \{-,0\} & \{-,0,+\} \\
			
			\{-,0\}	& \{0\} & \{-,0\}	& \{0,+\}	& \{0,+\}	& \{-,0,+\}	& \{-,0\} & \{-,0,+\} \\
			
			%\rowcolor[gray] {.8}
			\{-,+\} & \{0\} & \{-,+\}	& \{-,+\}	& \{-,0,+\}	& \{-,+\}	& \{-,0,+\} & \{-,0,+\} \\
			
			\{0,+\}	& \{0\} & \{0,+\}	& \{-,0\}	& \{-,0\}	& \{-,0,+\}	& \{0,+\} & \{-,0,+\} \\
			
			%\rowcolor[gray] {.8}
			\{-,0,+\} & \{0\} & \{-,0,+\} & \{-,0,+\} & \{-,0,+\} & \{-,0,+\} & \{-,0,+\} & \{-,0,+\} \\
		\end{tabular}
	\end{table}


	\begin{table}[ht!]
		\caption{Complete table of the division operation on the abstract domain $\{-,0,+\}$.}\label{tab:division}
		\begin{tabular}{>{\columncolor[gray]{0.8}}l||l|l|l|l|l|l|l}
			\rowcolor[gray] {.8}
			$/^\#$	& \{0\}			& \{+\}		& \{-\}		& \{-,0\}	& \{-,+\}	& \{0,+\} & \{-,0,+\} \\\hline\hline
			%\rowcolor[gray] {.8}
			\{0\}	& \{undefined\}	& \{0\}		& \{0\}		& \{undefined,0\}		& \{0\}		& \{undefined,0\}	& \{undefined,0\} \\
			
			\{+\} 	& \{undefined\}	& \{0,+\}	& \{-,0\}	& \{undefined,-,0\}		& \{-,0,+\}	& \{undefined,0,+\} & \{undefined,-,0,+\} \\
			
			%\rowcolor[gray] {.8}
			\{-\}	& \{undefined\}	& \{-,0\}	& \{0,+\}	& \{undefined,0,+\}		& \{-,0,+\}	& \{undefined,0,-\} & \{undefined,-,0,+\} \\
			
			\{-,0\}	& \{undefined\} & \{-,0\}	& \{0,+\}	& \{undefined,0,+\}		& \{-,0,+\}	& \{undefined,0,-\} & \{undefined,-,0,+\} \\
			
			%\rowcolor[gray] {.8}
			\{-,+\} & \{undefined\} & \{-,0,+\}	& \{-,0,+\}	& \{undefined,-,0,+\}	& \{-,0,+\}	& \{undefined,-,0,+\} & \{undefined,-,0,+\} \\
			
			\{0,+\}	& \{undefined\} & \{0,+\}	& \{-,0\}	& \{undefined,-,0\}		& \{-,0,+\}	& \{undefined,0,+\} & \{undefined,-,0,+\} \\
			
			%\rowcolor[gray] {.8}
			\{-,0,+\} & \{undefined\} & \{-,0,+\} & \{-,0,+\} & \{undefined,-,0,+\}	& \{-,0,+\}	& \{undefined,-,0,+\} & \{undefined,-,0,+\} \\
		\end{tabular}
	\end{table}
	
	
	
	\section*{Assignment 4.3}
	
	\begin{align*}
	S[0] &\supseteq \bot\\
	S[1] &\supseteq S[0] \oplus \{x \to \{+\}\}\\
	S[2] &\supseteq S[1] \oplus \{y \to \{+\}\}\\
	S[2] &\supseteq S[5]\\
	S[3] &\supseteq S[2]\\
	S[4] &\supseteq S[3] \oplus \{y\to[[x\times y]]^\# \, S[3]\}\\
	\textrm{For S[5]}& \textrm{ next one I'm not sure, it could be either of the following two:}\\
	S[5] &\supseteq S[4] \oplus \{x\to[[x-1]]^\# \, S[4]\}\\
	S[5] &\supseteq S[4] \oplus \{x\to\{-,0,+\}^\# \, S[4]\}\\\\
	S[6] &\supseteq S[2]\\
	S[7] &\supseteq S[6]
	\end{align*}
	\par which means that $S[7] = \{x \to \{-,0,+\}, y \to \{-,0,+\} \}$. 
	
	\begin{table}[ht!]
		\caption{Complete table of where in the abstract domain $\{-,0,+\}$ x and y are.}\label{tab:division}
		\hspace*{-2.65cm}\begin{tabular}{l l|l l|l l|l l|l l|l l|l l|l l}
			\rowcolor[gray] {.8}
			0&	&	1&	&	2&	&	3&	&	4&	&	5&	&	6&	&	7& \\	
			x&	y&	x&	y&	x&	y&	x&	y&	x&	y&	x&	y&	x&	y&	x&	y \\
			\{\}&	\{\}&	\{+\}&	\{\}&	\{+\}&	\{+\}&	\{+\}&	\{+\}&	\{+\}&	\{+\}&	\{-,0,+\}&	\{+\}&	\{\}&	\{\}&	\{\}&	\{\}\\
			\{\}&	\{\}&	\{+\}&	\{\}&	\{-,0,+\}&	\{+\}&	\{-,0,+\}&	\{+\}&	\{-,0,+\}&	\{-,0,+\}&	\{-,0,+\}&	\{-,0,+\}&	\{\}&	\{\}&	\{\}&	\{\}\\
			\{\}&	\{\}&	\{+\}&	\{\}&	\{-,0,+\}&	\{-,0,+\}&	\{-,0,+\}&	\{-,0,+\}&	\{-,0,+\}&	\{-,0,+\}&	\{-,0,+\}&	\{-,0,+\}&	\{\}&	\{\}&	\{\}&	\{\}\\
			\{\}&	\{\}&	\{+\}&	\{\}&	\{-,0,+\}&	\{-,0,+\}&	\{-,0,+\}&	\{-,0,+\}&	\{-,0,+\}&	\{-,0,+\}&	\{-,0,+\}&	\{-,0,+\}&	\{\}&	\{\}&	\{\}&	\{\}\\
			\{\}&	\{\}&	\{+\}&	\{\}&	\{-,0,+\}&	\{-,0,+\}&	\{-,0,+\}&	\{-,0,+\}&	\{-,0,+\}&	\{-,0,+\}&	\{-,0,+\}&	\{-,0,+\}&	\{\}&	\{\}&	\{\}&	\{\}\\
			\{\}&	\{\}&	\{+\}&	\{\}&	\{-,0,+\}&	\{-,0,+\}&	\{-,0,+\}&	\{-,0,+\}&	\{-,0,+\}&	\{-,0,+\}&	\{-,0,+\}&	\{-,0,+\}&	\{\}&	\{\}&	\{\}&	\{\}\\
			\{\}&	\{\}&	\{+\}&	\{\}&	\{-,0,+\}&	\{-,0,+\}&	\{-,0,+\}&	\{-,0,+\}&	\{-,0,+\}&	\{-,0,+\}&	\{-,0,+\}&	\{-,0,+\}&	\{\}&	\{\}&	\{\}&	\{\}\\
			\{\}&	\{\}&	\{+\}&	\{\}&	\{-,0,+\}&	\{-,0,+\}&	\{-,0,+\}&	\{-,0,+\}&	\{-,0,+\}&	\{-,0,+\}&	\{-,0,+\}&	\{-,0,+\}&	\{\}&	\{\}&	\{\}&	\{\}\\
			\{\}&	\{\}&	\{+\}&	\{\}&	\{-,0,+\}&	\{-,0,+\}&	\{-,0,+\}&	\{-,0,+\}&	\{-,0,+\}&	\{-,0,+\}&	\{-,0,+\}&	\{-,0,+\}&	\{\}&	\{\}&	\{\}&	\{\}\\
			\{\}&	\{\}&	\{+\}&	\{\}&	\{-,0,+\}&	\{-,0,+\}&	\{-,0,+\}&	\{-,0,+\}&	\{-,0,+\}&	\{-,0,+\}&	\{-,0,+\}&	\{-,0,+\}&	\{-,0,+\}&	\{-,0,+\}&	\{-,0,+\}&	\{-,0,+\}

		\end{tabular}
	\end{table}
	
	
	If anything more is required for this question, I apparently did not understand this question.
	
	
	\section*{Assignment 4.4}
	\par The domain for this assignment is $\{even, odd\} \supseteq \{even\}; \{even, odd\} \supseteq \{odd\}; \{even\} \supseteq \{\}; \{odd\} \supseteq \{\}$ (0 is even).
	\par Tables \ref{tab:addition.4} and \ref{tab:multiplication4.4} show the effects of the addition and multiplication operation on the domain, respectively.
	
	\begin{table}[ht!]
		\caption{Complete table of the addition operation on the abstract domain $\{even,odd\}$.}\label{tab:addition.4}
		\begin{tabular}{>{\columncolor[gray]{0.8}}l||l|l|l}
			\rowcolor[gray] {.8}
			$+^\#$		& \{even\}	& \{odd\}		& \{even,odd\}\\\hline\hline
			%\rowcolor[gray] {.8}
			\{even\}	& \{even\}	& \{odd\}		& \{even,odd\}		\\
			
			\{odd\}		& \{odd\}	& \{even\}		& \{even,odd\}		\\
			
			\{even,odd\}& \{even,odd\}	& \{even,odd\}	& \{even,odd\}	\\
			
		\end{tabular}
	\end{table}
	
	\begin{table}[ht!]
		\caption{Complete table of the multiplication operation on the abstract domain $\{even,odd\}$.}\label{tab:multiplication4.4}
		\begin{tabular}{>{\columncolor[gray]{0.8}}l||l|l|l}
			\rowcolor[gray] {.8}
			$*^\#$		& \{even\}	& \{odd\}		& \{even,odd\}\\\hline\hline
			%\rowcolor[gray] {.8}
			\{even\}	& \{even\}	& \{even\}		& \{even\}		\\
			
			\{odd\}		& \{even\}	& \{odd\}		& \{even,odd\}		\\
			
			\{even,odd\}& \{even\}	& \{even,odd\}	& \{even,odd\}	\\

		\end{tabular}
	\end{table}
	Edge effect for this domain are as follows:	
	\begin{align*}
	&[[;]]^\# \,D = D \\
	&[[true (e)]]^\#\,D = D\\
	&[[false (e)]]^\#\,D = D\\
	&[[x = e;]]^\#\,D = D \oplus \{x \to [[e]]^\#\,D\}\\
	&[[x = M [e];]]^\#\,D = D \oplus \{x\to \{even,odd\}\}\\
	&[[M [e_1] = e_2;]]^\#\,D = D\\
	&... whenever D \neq \bot
	\end{align*}
	\par If anything more is required to fully answer this question, I apparently did not fully understand this question.
	
	
	\newpage 
	\bibliographystyle{unsrtnat}
	%\bibliographystyle{authordate1}
	\bibliography{bibliography} % The file containing the bibliography
%	\newpage
	

%	\begin{figure}[ht!]
%		\hspace{-2mm}\makebox[\textwidth][c]{
%			\begin{tabular}{@{}p{0.5\linewidth}@{\hspace{5mm}}p{0.5\linewidth}@{}}
%				\subfigimg[width = 0.55\textwidth]{\hspace{0mm}(a)}{Offseteffect} &
%				\subfigimg[width = 0.55\textwidth]{\hspace{0mm}(b)}{RCeffect}
%		\end{tabular}}
%	\end{figure}
		
	
\end{document}
